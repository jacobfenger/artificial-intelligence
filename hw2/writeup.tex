\documentclass[22pt]{article}

\usepackage[margin=1in]{geometry}
\usepackage{amsmath,amsthm,amssymb, graphicx, listings, multicol, varwidth}

\begin{document}

\title{CSCE 625 Homework \#1}
\author{Jacob Fenger}
\date{\today}
\maketitle

\section{Sammy's Sport Shop}
\subsection*{a)}
A propositional knowledge base is shown below:

\begin{multicols}{2}
\begin{enumerate}
	\item $O1Y$
	\item $O2W$
	\item $O3Y$
	\item $L1W$
	\item $L2Y$
	\item $L3B$
	\item $O1Y \implies C1Y \lor C1B$ 
	\item $O2W \implies C2W \lor C2B$
	\item $O3Y \implies C3B \lor C3Y$
	\item $C1W \implies \neg C2W \land \neg C3W$
	\item $C1Y \implies \neg C2Y \land \neg C3Y$
	\item $C1B \implies \neg C2B \land \neg C3B$
	\item $C2W \implies \neg C1W \land \neg C3W$
	\item $C2Y \implies \neg C1Y \land \neg C3Y$
	\item $C2B \implies \neg C1B \land \neg C3B$
	\item $C3W \implies \neg C1W \land \neg C2W$
	\item $C3Y \implies \neg C1Y \land \neg C2Y$
	\item $C3B \implies \neg C1B \land \neg C2B$
	\item $L1Y \implies \neg C1Y$
	\item $L1W \implies \neg C1W$
	\item $L1B \implies \neg C1B$
	\item $L2Y \implies \neg C2Y$
	\item $L2W \implies \neg C2W$
	\item $L2B \implies \neg C2B$
	\item $L3Y \implies \neg C3Y$
	\item $L3W \implies \neg C3W$
	\item $L3B \implies \neg C3B$
	\item $C1W \lor C1Y \lor C1B$
	\item $C2W \lor C2Y \lor C2B$
	\item $C3W \lor C3Y \lor C3B$
\end{enumerate}
\end{multicols}

\subsection*{b)}
The following are the steps following via Natural Deduction to prove that 
box 2 must contain white balls ($C2W$).

\begin{multicols}{2}
\begin{enumerate}
	\setcounter{enumi}{30}
	\item $\neg C3B$ (M.P. on 27)
	\item $C3B \lor C3Y$ (M.P. on 9)
	\item $C3Y$ (Res. on 31 and 32)
	\item $\neg C1Y \land \neg C2Y$ (M.P. on 17)
	\item $C1Y \lor C1B$ (M.P. on 7)
	\item $C2W \lor C2B$ (M.P. on 8)
	\item $\neg C1Y$ (Conj. Elimination on 34)
	\item $\neg C2Y$ (Conj. elimination on 34)
	\item $C1B$ (Res. on 35 and 37)
	\item $\neg C2B \land \neg C3B$ (M.P on 12)
	\item $\neg C2B$ (And elimination on 40)
	\item $\neg C3B$ (And elimination on 40)
	\item $C2W$ (Res. on 36 and 41)
\end{enumerate}
\end{multicols}

\subsection*{c)}
First we must convert the propositional knowledge base to CNF:

\begin{multicols}{2}
\begin{enumerate}
	\item $O1Y$
	\item $O2W$
	\item $O3Y$
	\item $L1W$
	\item $L2Y$
	\item $L3B$
	\item $\neg O1Y \lor C1Y \lor C1B$
	\item $\neg O2W \lor C2W \lor C2B$
	\item $\neg O3Y \lor C3B \lor C3Y$

	\item
	\begin{enumerate}
		\item $\neg C1W \lor \neg C2W$
		\item $\neg C1W \lor \neg C3W$
	\end{enumerate}

	\item
	\begin{enumerate}
		\item $\neg C1Y \lor \neg C2Y$
		\item $\neg C1Y \lor \neg C3Y$
	\end{enumerate}

	\item
	\begin{enumerate}

		\item $\neg C1B \lor \neg C2B$
		\item $\neg C1B \lor \neg C3B$
	\end{enumerate}

	\item
	\begin{enumerate}
		\item $\neg C2W \lor \neg C1W$
		\item $\neg C2W \lor \neg C3W$
	\end{enumerate}
	
	\item
	\begin{enumerate}
		\item $\neg C2Y \lor \neg C1Y$
		\item $\neg C2Y \lor \neg C3Y$
	\end{enumerate}

	\item
	\begin{enumerate}
		\item $\neg C2B \lor \neg C1B$
		\item $\neg C2B \lor \neg C3B$
	\end{enumerate}

	\item
	\begin{enumerate}
		\item $\neg C3W \lor \neg C1W$
		\item $\neg C3W \lor \neg C2W$
	\end{enumerate}

	\item
	\begin{enumerate}
		\item $\neg C3Y \lor \neg C1Y$
		\item $\neg C3Y \lor \neg C2Y$
	\end{enumerate}

	\item
	\begin{enumerate}
		\item $\neg C3B \lor \neg C1B$
		\item $\neg C3B \lor \neg C2B$
	\end{enumerate}

	\item $\neg L1Y \lor \neg C1Y$
	\item $\neg L1W \lor \neg C1W$
	\item $\neg L1B \lor \neg C1B$
	\item $\neg L2Y \lor \neg C2Y$
	\item $\neg L2W \lor \neg C2W$
	\item $\neg L2B \lor \neg C2B$
	\item $\neg L3Y \lor \neg C3Y$
	\item $\neg L3W \lor \neg C3W$
	\item $\neg L3B \lor \neg C3B$
\end{enumerate}
\end{multicols}

The following steps show the process of resolution refutation (Note: 
Some of the useless steps when finding a solution were left out):

\begin{multicols}{2}
\begin{enumerate}
	\setcounter{enumi}{27}
	\item $\neg C2W$ (Assume negation of query)
	\item $C1Y \lor C1B$ (Res. on 1 and 7)
	\item $C2W \lor C2B$ (Res. on 2 and 8)
	\item $C3B \lor C3Y$ (Res. on 3 and 9)
	\item $\neg C3B$ (Res on 6 and 27)
	\item $C3Y$ (Res. on 31 and 32)
	\item $\neg C1Y$ (Res. on 11b and 33)
	\item $C1B$ (Res. on 29 and 34)
	\item $\neg C2B$ (Res on 12a and 35)
	\item $C2W$ (Res. on 30 and 36)
	\item $\square$ (End of proof since empty clause was found)
\end{enumerate}
\end{multicols}

\newpage

\section{4-Queens Problem}

$Q_{i,j}$ represents a queen in row $i$ and column $j$. This notation will be 
used heavily throughout this problem. \\

The knowledge base in CNF is shown below :

\begin{multicols}{4}
\begin{enumerate}
	% Every row has at least one queen
	\item $Q_{1,1} \lor Q_{1,2} \lor Q_{1,3} \lor Q_{1,4}$
	\item $Q_{2,1} \lor Q_{2,2} \lor Q_{2,3} \lor Q_{2,4}$
	\item $Q_{3,1} \lor Q_{3,2} \lor Q_{3,3} \lor Q_{3,4}$
	\item $Q_{4,1} \lor Q_{4,2} \lor Q_{4,3} \lor Q_{4,4}$

	% Row 1 has at most 1 queen
	\item $\neg Q_{1,1} \lor \neg Q_{1,2}$
	\item $\neg Q_{1,1} \lor \neg Q_{1,3}$
	\item $\neg Q_{1,1} \lor \neg Q_{1,4}$
	\item $\neg Q_{1,2} \lor \neg Q_{1,3}$
	\item $\neg Q_{1,2} \lor \neg Q_{1,4}$
	\item $\neg Q_{1,3} \lor \neg Q_{1,4}$

	% Row 2 has at most 1 queen 
	\item $\neg Q_{2,1} \lor \neg Q_{2,2}$
	\item $\neg Q_{2,1} \lor \neg Q_{2,3}$
	\item $\neg Q_{2,1} \lor \neg Q_{2,4}$
	\item $\neg Q_{2,2} \lor \neg Q_{2,3}$
	\item $\neg Q_{2,2} \lor \neg Q_{2,4}$
	\item $\neg Q_{2,3} \lor \neg Q_{2,4}$

	% Row 3 has at most 1 queen
	\item $\neg Q_{3,1} \lor \neg Q_{3,2}$
	\item $\neg Q_{3,1} \lor \neg Q_{3,3}$
	\item $\neg Q_{3,1} \lor \neg Q_{3,4}$
	\item $\neg Q_{3,2} \lor \neg Q_{3,3}$
	\item $\neg Q_{3,2} \lor \neg Q_{3,4}$
	\item $\neg Q_{3,3} \lor \neg Q_{3,4}$

	% Row 4 has at most 1 queen
	\item $\neg Q_{4,1} \lor \neg Q_{4,2}$
	\item $\neg Q_{4,1} \lor \neg Q_{4,3}$
	\item $\neg Q_{4,1} \lor \neg Q_{4,4}$
	\item $\neg Q_{4,2} \lor \neg Q_{4,3}$
	\item $\neg Q_{4,2} \lor \neg Q_{4,4}$
	\item $\neg Q_{4,3} \lor \neg Q_{4,4}$

	% Column 1 has at most 1 queen
	\item $\neg Q_{1,1} \lor \neg Q_{2,1}$
	\item $\neg Q_{1,1} \lor \neg Q_{3,1}$
	\item $\neg Q_{1,1} \lor \neg Q_{4,1}$
	\item $\neg Q_{2,1} \lor \neg Q_{3,1}$
	\item $\neg Q_{2,1} \lor \neg Q_{4,1}$
	\item $\neg Q_{3,1} \lor \neg Q_{4,1}$

	% Column 2 has at most 1 queen
	\item $\neg Q_{1,2} \lor \neg Q_{2,2}$
	\item $\neg Q_{1,2} \lor \neg Q_{3,2}$
	\item $\neg Q_{1,2} \lor \neg Q_{4,2}$
	\item $\neg Q_{2,2} \lor \neg Q_{3,2}$
	\item $\neg Q_{2,2} \lor \neg Q_{4,2}$
	\item $\neg Q_{3,2} \lor \neg Q_{4,2}$

	% Column 3 has at most 1 queen
	\item $\neg Q_{1,3} \lor \neg Q_{2,3}$
	\item $\neg Q_{1,3} \lor \neg Q_{3,3}$
	\item $\neg Q_{1,3} \lor \neg Q_{4,3}$
	\item $\neg Q_{2,3} \lor \neg Q_{3,3}$
	\item $\neg Q_{2,3} \lor \neg Q_{4,3}$
	\item $\neg Q_{3,3} \lor \neg Q_{4,3}$

	% Column 4 has at most 1 queen
	\item $\neg Q_{1,4} \lor \neg Q_{2,4}$
	\item $\neg Q_{1,4} \lor \neg Q_{3,4}$
	\item $\neg Q_{1,4} \lor \neg Q_{4,4}$
	\item $\neg Q_{2,4} \lor \neg Q_{3,4}$
	\item $\neg Q_{2,4} \lor \neg Q_{4,4}$
	\item $\neg Q_{3,4} \lor \neg Q_{4,4}$

	% Diagonals for column 1
	\item $\neg Q_{1,1} \lor \neg Q_{2,2}$
	\item $\neg Q_{1,1} \lor \neg Q_{3,3}$
	\item $\neg Q_{1,1} \lor \neg Q_{4,4}$
	\item $\neg Q_{2,1} \lor \neg Q_{3,2}$
	\item $\neg Q_{2,1} \lor \neg Q_{4,3}$
	\item $\neg Q_{2,1} \lor \neg Q_{1,2}$
	\item $\neg Q_{3,1} \lor \neg Q_{4,2}$
	\item $\neg Q_{3,1} \lor \neg Q_{2,2}$
	\item $\neg Q_{3,1} \lor \neg Q_{1,3}$
	\item $\neg Q_{4,1} \lor \neg Q_{3,2}$
	\item $\neg Q_{4,1} \lor \neg Q_{2,3}$
	\item $\neg Q_{4,1} \lor \neg Q_{1,4}$

	% Diagonals for column 2
	\item $\neg Q_{1,2} \lor \neg Q_{2,3}$
	\item $\neg Q_{1,2} \lor \neg Q_{3,4}$
	\item $\neg Q_{2,2} \lor \neg Q_{1,3}$
	\item $\neg Q_{2,2} \lor \neg Q_{3,3}$
	\item $\neg Q_{2,2} \lor \neg Q_{4,4}$
	\item $\neg Q_{3,2} \lor \neg Q_{4,3}$
	\item $\neg Q_{3,2} \lor \neg Q_{2,3}$
	\item $\neg Q_{3,2} \lor \neg Q_{1,4}$
	\item $\neg Q_{4,2} \lor \neg Q_{3,3}$
	\item $\neg Q_{4,2} \lor \neg Q_{2,4}$

	% Diagonals for column 3
	\item $\neg Q_{1,3} \lor \neg Q_{2,4}$
	\item $\neg Q_{2,3} \lor \neg Q_{3,2}$
	\item $\neg Q_{2,3} \lor \neg Q_{3,4}$
	\item $\neg Q_{2,3} \lor \neg Q_{1,4}$
	\item $\neg Q_{3,3} \lor \neg Q_{2,4}$
	\item $\neg Q_{3,3} \lor \neg Q_{4,4}$
	\item $\neg Q_{4,3} \lor \neg Q_{3,4}$

\end{enumerate}
\end{multicols}

\subsection*{a)}
The steps of using DPLL using no heuristics are shown below. Some 
steps were combined together due to readability and conciseness.
Final values are represented in the last row of the table.

\begin{table}[htbp]
\centering
\small
\setlength\tabcolsep{3pt}
\begin{tabular}{ |c|c|c|c|c|c|c|c|c|c|c|c|c|c|c|c|c| } 
 \hline
 $Q_{1,1}$ & $Q_{1,2}$ & $Q_{1,3}$ & $Q_{1,4}$ & $Q_{2,1}$ & $Q_{2,2}$ & $Q_{2,3}$ & $Q_{2,4}$ & $Q_{3,1}$ &
  $Q_{3,2}$ & $Q_{3,3}$ & $Q_{3,4}$ & $Q_{4,1}$ & $Q_{4,2}$ & $Q_{4,3}$ & $Q_{4,4}$ & Comments\\ \hline

 T & ? & ? & ? & ? & ? & ? & ? & ? & ? & ? & ? & ? & ? & ? & ? & Doesn't satisfy all reqs \\ \hline
 T & T & ? & ? & ? & ? & ? & ? & ? & ? & ? & ? & ? & ? & ? & ? & Violation on \#5 \\ \hline
 T & F & ? & ? & ? & ? & ? & ? & ? & ? & ? & ? & ? & ? & ? & ? & No violation, reqs. not satisfied \\ \hline
 T & F & T & ? & ? & ? & ? & ? & ? & ? & ? & ? & ? & ? & ? & ? & Violation on \#6 \\ \hline
 T & T & F & ? & ? & ? & ? & ? & ? & ? & ? & ? & ? & ? & ? & ? & No violation, reqs. not satisfied \\ \hline
 T & F & F & T & ? & ? & ? & ? & ? & ? & ? & ? & ? & ? & ? & ? & Violation on \#7 \\ \hline
 T & F & F & F & ? & ? & ? & ? & ? & ? & ? & ? & ? & ? & ? & ? & No violation, reqs. not satisfied \\ \hline
 T & F & F & F & T & ? & ? & ? & ? & ? & ? & ? & ? & ? & ? & ? & Violation on \#29 \\ \hline
 T & F & F & F & F & T & ? & ? & ? & ? & ? & ? & ? & ? & ? & ? & Violation on \#53 \\ \hline
 T & F & F & F & F & F & T & ? & ? & ? & ? & ? & ? & ? & ? & ? & Continue \\ \hline
 T & F & F & F & F & F & T & F & T & ? & ? & ? & ? & ? & ? & ? & Violation on \#30 \\ \hline
 T & F & F & F & F & F & T & F & F & T & ? & ? & ? & ? & ? & ? & Violation on \#71 \\ \hline
 T & F & F & F & F & F & T & F & F & F & T & ? & ? & ? & ? & ? & Violation on \#44 \\ \hline
 T & F & F & F & F & F & T & F & F & F & F & T & ? & ? & ? & ? & Violation on \#77 \\ \hline
 T & F & F & F & F & F & T & F & F & F & F & F & ? & ? & ? & ? & Violation on \#3 (Backtrack) \\ \hline
 T & F & F & F & F & F & F & T & ? & ? & ? & ? & ? & ? & ? & ? & Continue \\ \hline
 T & F & F & F & F & F & F & T & T & ? & ? & ? & ? & ? & ? & ? & Violation on \#30 \\ \hline
 T & F & F & F & F & F & F & T & F & T & ? & ? & ? & ? & ? & ? & Continue \\ \hline
 T & F & F & F & F & F & F & T & F & T & T & ? & ? & ? & ? & ? & Violation on \#20 \\ \hline
 T & F & F & F & F & F & F & T & F & T & F & T & ? & ? & ? & ? & Violation on \#21 \\ \hline
 T & F & F & F & F & F & F & T & F & T & F & F & T & ? & ? & ? & Violation on \#31 \\ \hline
 T & F & F & F & F & F & F & T & F & T & F & F & F & T & ? & ? & Violation on \#40 \\ \hline
 T & F & F & F & F & F & F & T & F & T & F & F & F & F & T & ? & Violation on \#70 \\ \hline
 T & F & F & F & F & F & F & T & F & T & F & F & F & F & F & T & Violation on \#55 \\ \hline
 T & F & F & F & F & F & F & T & F & T & F & F & F & F & F & F & Violation on \#4 (Backtrack) \\ \hline
 T & F & F & F & F & F & F & T & F & F & T & ? & ? & ? & ? & ? & Violation on \#54 \\ \hline
 T & F & F & F & F & F & F & T & F & F & F & T & ? & ? & ? & ? & Violation on \#50 \\ \hline
 T & F & F & F & F & F & F & T & F & F & F & F & ? & ? & ? & ? & Violation on \#3 (Backtrack) \\ \hline
 T & F & F & F & F & F & F & F & ? & ? & ? & ? & ? & ? & ? & ? & Violation on \#2 (Backtrack) \\ \hline
 F & T & ? & ? & ? & ? & ? & ? & ? & ? & ? & ? & ? & ? & ? & ? & Continue \\ \hline
 F & T & T & ? & ? & ? & ? & ? & ? & ? & ? & ? & ? & ? & ? & ? & Violation on \#10 \\ \hline
 F & T & F & T & ? & ? & ? & ? & ? & ? & ? & ? & ? & ? & ? & ? & Violation on \#9 \\ \hline
 F & T & F & F & ? & ? & ? & ? & ? & ? & ? & ? & ? & ? & ? & ? & Continue \\ \hline
 F & T & F & F & T & ? & ? & ? & ? & ? & ? & ? & ? & ? & ? & ? & Violation on \#58 \\ \hline
 F & T & F & F & F & T & ? & ? & ? & ? & ? & ? & ? & ? & ? & ? & Violation on \#35 \\ \hline
 F & T & F & F & F & F & T & ? & ? & ? & ? & ? & ? & ? & ? & ? & Violation on \#65 \\ \hline
 F & T & F & F & F & F & F & T & ? & ? & ? & ? & ? & ? & ? & ? & Continue \\ \hline
 F & T & F & F & F & F & F & T & T & F & ? & ? & ? & ? & ? & ? & Continue \\ \hline
 F & T & F & F & F & F & F & T & T & F & F & ? & ? & ? & ? & ? & Continue \\ \hline
 F & T & F & F & F & F & F & T & T & F & F & F & ? & ? & ? & ? & Continue \\ \hline
 F & T & F & F & F & F & F & T & T & F & F & F & T & ? & ? & ? & Violation on \#34 \\ \hline
 F & T & F & F & F & F & F & T & T & F & F & F & F & T & ? & ? & Violation on \#59 \\ \hline
 F & T & F & F & F & F & F & T & T & F & F & F & F & F & T & ? & Continue \\ \hline
 F & T & F & F & F & F & F & T & T & F & F & F & F & F & T & F & Done. All clauses satisfied. \\ \hline
 \end{tabular}
 \end{table}


 \newpage

 \subsection*{b)}

 The steps of DPLL using PureSymbol and UnitClause heuristics are shown below in the table:

 \begin{table}[htbp]
\centering
\small
\setlength\tabcolsep{3pt}
\begin{tabular}{ |c|c|c|c|c|c|c|c|c|c|c|c|c|c|c|c|c| } 
 \hline
 $Q_{1,1}$ & $Q_{1,2}$ & $Q_{1,3}$ & $Q_{1,4}$ & $Q_{2,1}$ & $Q_{2,2}$ & $Q_{2,3}$ & $Q_{2,4}$ & $Q_{3,1}$ &
  $Q_{3,2}$ & $Q_{3,3}$ & $Q_{3,4}$ & $Q_{4,1}$ & $Q_{4,2}$ & $Q_{4,3}$ & $Q_{4,4}$ & Comments\\ \hline

 T & ? & ? & ? & ? & ? & ? & ? & ? & ? & ? & ? & ? & ? & ? & ? & Doesn't satisfy all reqs (Continue) \\ \hline
 T & F & F & F & F & F & ? & ? & F & ? & F & ? & F & ? & ? & F & U.C. on \#5, 6, 7, 29, 30, 31, 53, 54, 55 \\ \hline
 T & F & F & F & F & F & T & ? & F & ? & F & ? & F & ? & ? & F & Continue \\ \hline
 T & F & F & F & F & F & T & F & F & ? & F & ? & F & ? & ? & F & U.C. on \#16 \\ \hline
 T & F & F & F & F & F & T & F & F & F & F & ? & F & ? & F & F & U.C. on \#45, 76 \\ \hline
 T & F & F & F & F & F & T & F & F & F & F & F & F & ? & F & F & U.C. on \#77, violation on \#3 \\ \hline
 T & F & F & F & F & F & F & T & F & ? & F & ? & F & ? & ? & F & Backtrack, Continue \\ \hline
 T & F & F & F & F & F & F & T & F & ? & F & F & F & ? & ? & F & U.C. on \#50 \\ \hline
 T & F & F & F & F & F & F & T & F & T & F & F & F & ? & ? & F & U.C. on \#3 \\ \hline
 T & F & F & F & F & F & F & T & F & T & F & F & F & F & ? & F & U.C. on \#40 \\ \hline
 T & F & F & F & F & F & F & T & F & T & F & F & F & F & F & F & U.C. on \#70, violation \#4 \\ \hline
 T & F & F & F & F & F & F & F & F & ? & F & ? & F & ? & ? & F & Backtrack, Continue \\ \hline
 T & F & F & F & F & F & F & F & F & ? & F & ? & F & ? & ? & F & Violation on \#2 (Backtrack) \\ \hline
 F & T & F & F & ? & ? & ? & ? & ? & ? & ? & ? & ? & ? & ? & ? & U.C. on \#8, 9 \\ \hline
 F & T & F & F & F & F & F & ? & ? & F & ? & F & ? & F & ? & ? & U.C. on \#35, 36, 37, 58, 65, 66 \\ \hline
 F & T & F & F & F & F & F & T & ? & F & ? & F & ? & F & ? & ? & U.C. on \#2 \\ \hline
 F & T & F & F & F & F & F & T & ? & F & F & F & ? & F & ? & F & U.C. on \#50, 51 \\ \hline
 F & T & F & F & F & F & F & T & T & F & F & F & ? & F & ? & F & U.C. on \#3 \\ \hline
 F & T & F & F & F & F & F & T & T & F & F & F & F & F & ? & F & U.C. on \#34 \\ \hline
 F & T & F & F & F & F & F & T & T & F & F & F & F & F & T & F & U.C. on \#4 \\ \hline
 F & T & F & F & F & F & F & T & T & F & F & F & F & F & T & F & No violations! Search over! \\ \hline
 \end{tabular}
 \end{table}

Final values are represented in the very last row of the table.

\newpage

\section{Tic-Tac-Toe}

For the first set of rules, we have rules that state whenever X
can win, we should make that move. This might be obvious, but 
needs to be defined in the knowledge base.

\begin{multicols}{2}
\begin{enumerate}
	\item $CanWinX_{11} \implies MoveX_{11}$
	\item $CanWinX_{12} \implies MoveX_{12}$
	\item $CanWinX_{13} \implies MoveX_{13}$
	\item $CanWinX_{21} \implies MoveX_{21}$
	\item $CanWinX_{22} \implies MoveX_{22}$
	\item $CanWinX_{23} \implies MoveX_{23}$
	\item $CanWinX_{31} \implies MoveX_{31}$
	\item $CanWinX_{32} \implies MoveX_{32}$
	\item $CanWinX_{33} \implies MoveX_{33}$
\end{enumerate}
\end{multicols}

\noindent Next, we need to define when $CanWinX$ for all positions. This is true 
when there are two positions with $X$ in them that can make a line (Vertically,
horizontally, or diagonally). We do not care about the rest of the pieces on 
the board since $X$ can win in the current turn.

\begin{multicols}{2}
\begin{enumerate}
	\item $\{?_{11} \land X_{22} \land X_{33}\} \implies CanWinX_{11}$
	\item $\{?_{11} \land X_{12} \land X_{13}\} \implies CanWinX_{11}$
	\item $\{?_{11} \land X_{21} \land X_{31}\} \implies CanWinX_{11}$

	\item $\{X_{11} \land ?_{12} \land X_{13}\} \implies CanWinX_{12}$
	\item $\{?_{12} \land X_{22} \land X_{32}\} \implies CanWinX_{12}$

	\item $\{X_{11} \land X_{12} \land ?_{13}\} \implies CanWinX_{13}$
	\item $\{?_{13} \land X_{22} \land X_{31}\} \implies CanWinX_{13}$
	\item $\{?_{13} \land X_{23} \land ?_{33}\} \implies CanWinX_{13}$

	\item $\{X_{11} \land ?_{21} \land X_{31}\} \implies CanWinX_{21}$
	\item $\{?_{21} \land X_{22} \land X_{23}\} \implies CanWinX_{21}$

	\item $\{X_{11} \land ?_{22} \land X_{33}\} \implies CanWinX_{22}$
	\item $\{X_{12} \land ?_{22} \land X_{32}\} \implies CanWinX_{22}$
	\item $\{X_{13} \land ?_{22} \land X_{31}\} \implies CanWinX_{22}$
	\item $\{X_{21} \land ?_{22} \land X_{23}\} \implies CanWinX_{22}$

	\item $\{X_{21} \land X_{22} \land ?_{23}\} \implies CanWinX_{23}$
	\item $\{X_{13} \land ?_{23} \land X_{33}\} \implies CanWinX_{23}$

	\item $\{X_{11} \land X_{21} \land ?_{31}\} \implies CanWinX_{31}$
	\item $\{X_{13} \land X_{22} \land ?_{31}\} \implies CanWinX_{31}$
	\item $\{?_{31} \land X_{32} \land X_{33}\} \implies CanWinX_{31}$

	\item $\{X_{31} \land ?_{32} \land X_{33}\} \implies CanWinX_{32}$
	\item $\{X_{12} \land X_{22} \land ?_{32}\} \implies CanWinX_{32}$

	\item $\{X_{11} \land X_{22} \land ?_{33}\} \implies CanWinX_{33}$
	\item $\{X_{13} \land X_{23} \land ?_{33}\} \implies CanWinX_{33}$
	\item $\{X_{31} \land X_{32} \land ?_{33}\} \implies CanWinX_{33}$
\end{enumerate}
\end{multicols}

\noindent Now we must define all the possible moves that are necessary because 
of a force move:

\begin{multicols}{2}
\begin{enumerate}
	\item $ForcedMoveX_{11} \implies MoveX_{11}$
	\item $ForcedMoveX_{12} \implies MoveX_{12}$
	\item $ForcedMoveX_{13} \implies MoveX_{13}$
	\item $ForcedMoveX_{21} \implies MoveX_{21}$
	\item $ForcedMoveX_{22} \implies MoveX_{22}$
	\item $ForcedMoveX_{23} \implies MoveX_{23}$
	\item $ForcedMoveX_{31} \implies MoveX_{31}$
	\item $ForcedMoveX_{32} \implies MoveX_{32}$
	\item $ForcedMoveX_{33} \implies MoveX_{33}$
\end{enumerate}
\end{multicols}

\noindent To help with readability, we define a rule for our knowledge base that is true
when no possible moves for X will result in a win for X:

$$\neg CanWinX_{11} \land CanWinX_{12} \land CanWinX_{13} \land CanWinX_{21} \land $$
$$ CanWinX_{22} \land CanWinX_{23} \land CanWinX_{31} \land CanWinX_{32} \land CanWinX_{33} $$
$$ \implies CannotWinX $$

\noindent Using the above rule and the possible winning conditions for $O$ (The other player), we can 
write rules for our knowledge base that state that if $O$ can win and $X$ cannot win in 
the current move, we must make a forced move. This will lead to a priority of winning moves
over forced move for $X$.

\begin{enumerate}
	\item $CanWinO_{11} \land CannotWinX \implies ForcedMoveX_{11}$
	\item $CanWinO_{12} \land CannotWinX \implies ForcedMoveX_{12}$
	\item $CanWinO_{13} \land CannotWinX \implies ForcedMoveX_{13}$
	\item $CanWinO_{21} \land CannotWinX \implies ForcedMoveX_{21}$
	\item $CanWinO_{22} \land CannotWinX \implies ForcedMoveX_{22}$
	\item $CanWinO_{23} \land CannotWinX \implies ForcedMoveX_{23}$
	\item $CanWinO_{31} \land CannotWinX \implies ForcedMoveX_{31}$
	\item $CanWinO_{32} \land CannotWinX \implies ForcedMoveX_{32}$
	\item $CanWinO_{33} \land CannotWinX \implies ForcedMoveX_{33}$
\end{enumerate}

\noindent We then must define winning conditions for $O$. This is exactly the same 
as the set of winning conditions for $X$ defined above.

\begin{multicols}{2}
\begin{enumerate}
	\item $\{?_{11} \land O_{22} \land O_{33}\} \implies CanWinO_{11}$
	\item $\{?_{11} \land O_{12} \land O_{13}\} \implies CanWinO_{11}$
	\item $\{?_{11} \land O_{21} \land O_{31}\} \implies CanWinO_{11}$

	\item $\{O_{11} \land ?_{12} \land O_{13}\} \implies CanWinO_{12}$
	\item $\{?_{12} \land O_{22} \land O_{32}\} \implies CanWinO_{12}$

	\item $\{O_{11} \land O_{12} \land ?_{13}\} \implies CanWinO_{13}$
	\item $\{?_{13} \land O_{22} \land O_{31}\} \implies CanWinO_{13}$
	\item $\{?_{13} \land O_{23} \land ?_{33}\} \implies CanWinO_{13}$

	\item $\{O_{11} \land ?_{21} \land O_{31}\} \implies CanWinO_{21}$
	\item $\{?_{21} \land O_{22} \land O_{23}\} \implies CanWinO_{21}$

	\item $\{O_{11} \land ?_{22} \land O_{33}\} \implies CanWinO_{22}$
	\item $\{O_{12} \land ?_{22} \land O_{32}\} \implies CanWinO_{22}$
	\item $\{O_{13} \land ?_{22} \land O_{31}\} \implies CanWinO_{22}$
	\item $\{O_{21} \land ?_{22} \land O_{23}\} \implies CanWinO_{22}$

	\item $\{O_{21} \land O_{22} \land ?_{23}\} \implies CanWinO_{23}$
	\item $\{O_{13} \land ?_{23} \land O_{33}\} \implies CanWinO_{23}$

	\item $\{O_{11} \land O_{21} \land ?_{31}\} \implies CanWinO_{31}$
	\item $\{O_{13} \land O_{22} \land ?_{31}\} \implies CanWinO_{31}$
	\item $\{?_{31} \land O_{32} \land O_{33}\} \implies CanWinO_{31}$

	\item $\{O_{31} \land ?_{32} \land O_{33}\} \implies CanWinO_{32}$
	\item $\{O_{12} \land O_{22} \land ?_{32}\} \implies CanWinO_{32}$

	\item $\{O_{11} \land O_{22} \land ?_{33}\} \implies CanWinO_{33}$
	\item $\{O_{13} \land O_{23} \land ?_{33}\} \implies CanWinO_{33}$
	\item $\{O_{31} \land O_{32} \land ?_{33}\} \implies CanWinO_{33}$
\end{enumerate}
\end{multicols}

\end{document}