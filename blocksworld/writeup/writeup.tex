\documentclass[12pt]{article}

\usepackage[margin=1in]{geometry}
\usepackage{amsmath,amsthm,amssymb, graphicx, listings, minted}
\usepackage{placeins}

\lstset{language=C++}

% Janky stuff for figures below
\let\Oldsection\section
\renewcommand{\section}{\FloatBarrier\Oldsection}

\let\Oldsubsection\subsection
\renewcommand{\subsection}{\FloatBarrier\Oldsubsection}

\let\Oldsubsubsection\subsubsection
\renewcommand{\subsubsection}{\FloatBarrier\Oldsubsubsection}
% End of janky stuff 

\graphicspath{ {../images/} }

\begin{document}

\title{CSCE 625 Programming Assignment \#1}
\author{Jacob Fenger}
\date{\today}
\maketitle

\tableofcontents

\section{Running the Program}
To run the program:
\begin{enumerate}
	\item Compile blocksworld.cpp (I used g++ version 4.8.4). E.g. 'g++ blocksworld.cpp'.
	\item To run the program, you must specificy the stack and block number as command line 
	arguments (E.g. './a.out 3 5' for 3 stacks and 5 blocks).
	\item The output should now output information regarding the state of the program. 
	If a path is found to the goal state, it should show it as well as the number of iterations
	and the priority queue size. If no path is found, or the number of iterations exceeds 
	5000, then it will output a message regarding failure.
\end{enumerate}

\section{Example Traces}
The following shows an example trace of running with a stack and block size of 3 and 5
respectively.

\inputminted{text}{3-5-trace}

\section{Heuristic Description}
My heurestic gives the worst penalty when blocks are not in the correct spots in the 
final stack. The intuition here is that the algorithm will be more likely to remove blocks
from the final stack if they are not in the correct position so that it will allow the 
correct blocks to be placed there. The problem with doing just this is that the search 
will be more likely to place blocks into their correct spots even if the block below 
isn't in the correct position. This leads into the next component of my heuristic.

Blocks can be placed on top of lower valued blocks (Such as a B on top of an A) which will 
lead to more moves to get the lower block into the goal state. My heuristic penalizes 
states which have the lower block below other blocks when it is not in the goal state. 
This will ensure that states with lower blocks on top will be more likely to be chosen 
since less moves are required to reach the goal state in those situations.

\subsection{Admissability}
This heuristic is not admissable because the cost it estimates tends to always be 
higher than the actual cost of reaching the goal. In my opinion, it is very hard to find
an admissable heuristic in this problem without using a more extensive search algorithm.

\section{Performance Metrics}

Below is a table showing the average number of iterations, queue size, and path length
for solutions of different stack and block sizes. Each problem space was ran ten times
and averages were computed.

\begin{table}[ht]
\centering
\begin{tabular}{|c|c|c|c|c|c|}
\hline
	& \textbf{3-5} & \textbf{6-10} & \textbf{3-7} & \textbf{3-10} & \textbf{5-12} \\ \hline
\textbf{Iterations} & 22.9 & 27.67 & 106.6 & 528.8 & 64.4 \\ \hline
\textbf{Path Length} & 9.9 & 15.3 & 18.9 & 27 & 22 \\ \hline
\textbf{Queue Size} & 48.9 & 442.3 & 241.3 & 1367.8 &  716.2 \\ \hline
\textbf{Failures} & 0 & 4 & 0 & 0 & 5\\ \hline
\end{tabular}
\caption{Average iteration, average queue size, average path 
length, and number of failures out of 10 executions for 
solutions with stack-block sizes (3-5 represents three 
stacks, five blocks).}
\end{table}

\section{Results Discussion}

My heuristic worked fine for cases where the stack size was smaller regardless
of the number of blocks. As we scale the number of stacks up, I ended up with several 
failures depending on the starting state (I consider it a failure once my search hits 5000
iterations). This is because the number of neighbor states is much larger each iteration and my
heuristic does not do much to account for a lot of empty stacks.

I was able to solve 3 stacks and 12 blocks in a reasonable time for most starting states
which is the largest I tested. The queue grew at mostly a linear rate when compared to the 
number of iterations. 

When it comes to finding optimal solution, my program sometimes found it but only on the 
most simple starting states. My heuristic sometimes provided the same value for all 
the neighbors which causes my program to take a non-optimal route.

To improve my heuristic, I would need to find a way to make it a priority to place blocks 
with a lower value on top of blocks of a higher value at all times.

\end{document}